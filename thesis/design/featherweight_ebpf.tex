\subsection{Featherweight eBPF}
\label{subsec:featherweight_ebpf}
Featherweight-eBPF is an IMP-like language representing a small subset of the eBPF ISA.

Main restrictions:
\begin{itemize}
\item Only 64-bit values and operations. Operations are unsigned. 
  % \item No jumps or branching. Only straight-line programs.
  \item Only forward jumps are allowed in programs. No looping.
\item Along with a pointer to the program context (``some memory'') passed to the program in \texttt{r1}, the size of that memory that is valid to read and write to is also passed in \texttt{r2}. 
\end{itemize}


% TODO: Should this be in background?
At the beginning of execution of all programs, assume that
\begin{itemize}
\item \texttt{r1} contains \texttt{*ctx}, a pointer to "some memory".
\item \texttt{r2} contains a 64-bit integer indicating the size of the memory pointed to by \texttt{r1}, i.e. reading and writing bytes between \texttt{r1} and \texttt{r1+(r2-8)} are allowed
\item \texttt{r10} contains a frame pointer to 512 bytes of stack-space for the eBPF program
\item \texttt{r0}, \texttt{r3}, \texttt{r4}, \texttt{r5}, \texttt{r6}, \texttt{r7}, \texttt{r8} and \texttt{r9} are all "uninitialized", i.e. they have an unknown value at program start. In "proper eBPF" it is invalid to read an uninitialized register. In x86\_64 assembly, i.e. "jited eBPF", it is valid (but stupid in most cases).

\end{itemize}

At the end of all programs, \texttt{r0} will contain the return value. 
This means \texttt{r0} should be initialized before an exit-instruction is reached.

% Ensuring registers are initialized before they are read is a security policy, not a safety policy.
% Therefore it is not a part of the verification conditions I will generate. Instead the derivation rules will ensure a register read is valid, so programs violating this will not be ``well formed'' and will not produce a verification condition. 

\subsubsection{Defining FWeBPF}
\label{subsub:fw}

Primitives of FWeBPF are defined by the grammar
\begin{figure}[H]
  \centering
  \begin{tabular}{lclr}
    $p$ & $::=$ & $X$ & (A variable) \\
        & $|$ & $N$ & (An immediate value $\in 0..2^{32}-1$ ) \\
\end{tabular}    
\end{figure}


Expressions of FWeBPF are defined by the grammar
\begin{figure}[H]
  \centering
  \begin{tabular}{lclr}
    $e$ & $::=$ & $p$ & (A primitive) \\
        & $|$ & $p_1 \oplus  p_2$ & (Unsigned addition of two primitives with 64-bit wrapping) \\
      & $|$ & $p_1 \ominus  p_2$ & (Unsigned subtraction of two primitives with 64-bit wrapping) \\        
      & $|$ & $p_1 \otimes p_2$ & (Unsigned multiplication of two primitives with 64-bit wrapping) \\    
      & $|$ & $p_1 \oslash p_2$ & (Unsigned division of two primitives with 64-bit wrapping) \\
        & $|$ & $p_1 ~ \mathrm{xor} ~ p_2$ & (bitwise \texttt{xor} of two primitives with 64-bit wrapping) \\
        & $|$ & $p_1 ~ \mathrm{mod} ~ p_2$ & (unsigned modulo operation of two primitives with 64-bit wrapping) \\    
      & $|$ & $\mathrm{load(p_1,p_2)}$ & (memory load from base pointer $p_1$ with offset $p_2$) \\            
\end{tabular}    
\end{figure}

Expression Predicates of FWeBPF are defined by the grammar
\begin{figure}[H]
  \centering
  \begin{tabular}{lclr}
    $ep$ & $::=$ & \texttt{true} & \\
         & $|$ & $p = e$ & (Equality of a primitive and an expression) \\    
         & $|$ & $p \neq e$ & (Inequality of a primitive and an expression) \\
         & $|$ &  $p > e$ & (Greater than of a primitive and an expression) \\            
         % & $|$ &  $p \geq e$ & (Greater than or equal of a primitive and an expression) \\
         & $|$ &  $p < e$ & (Less than of a primitive and an expression) \\        
         & $|$ &  $p \leq e$ & (Less than or equal of a primitive and an expression) \\
\end{tabular}    
\end{figure}


Instructions of FWeBPF are defined by the grammar
\begin{figure}[H]
  \centering
  \begin{tabular}{lclr}
    $i$ & $::=$ & \texttt{exit} & (Exits the program with \texttt{r0} as return value) \\
        & $|$   & $x := e$      & (assignment) \\
        % & $|$   & $s_1 ; s_2$   & (sequence) \\
        & $|$   & $\mathrm{if} ~ ep ~ \mathrm{then} ~ idx$ & (conditional) \\
& $|$ & $\mathrm{store(dst, off, src)}$ & (Memory store operation) \\
\end{tabular}    
\end{figure}

% Statements of FWeBPF are defined by the grammar
% \begin{figure}[H]
%   \centering
%   \begin{tabular}{lclr}
%     $s$ & $::=$ & \texttt{exit} & (Exits the program with \texttt{r0} as return value) \\
%         & $|$   & $x := e$      & (assignment) \\
%         & $|$   & $s_1 ; s_2$   & (sequence) \\
%         & $|$   & $\mathrm{if} ~ ep ~ \mathrm{then} ~ s_1 ~ \mathrm{else} ~ s_2$ & (conditional) \\
% \end{tabular}    
% \end{figure}


A program is a 0-indexed vector of instructions $\mathcal{P}$, st. the index corresponds to the line number of the instruction $n$ minus 1, i.e. $\mathcal{P}_0$ corresponds to the first instruction of the program, $\mathcal{P}_1$ to the second and so on.

\subsubsection{Translating eBPF to FWeBPF}
The translation from eBPF assembly mnemonics to Featherweight eBPF are essentially rewrite rules. They can be seen in \ref{fig:translation}.

\begin{figure}[H]
  \centering
  \begin{tabular}{rcl}
    \textbf{eBPF assembly} & & \textbf{FWeBPF} \\
    \texttt{exit} & $=$ & \texttt{exit} \\
    $\mathrm{mov} ~ r_d, src$ & $=$ & $r_d := src$ \\
    $\mathrm{add} ~ r_d, src$ & $=$ & $r_d := r_d \oplus src$ \\
    $\mathrm{mul} ~ r_d, src$ & $=$ & $r_d := r_d \otimes src$ \\        
    $\mathrm{div} ~ r_d, src$ & $=$ & $r_d := r_d \oslash src$ \\
    $\mathrm{mod} ~ r_d, src$ & $=$ & $r_d := r_d ~ \mathrm{mod} ~ src$ \\
    $\mathrm{xor} ~ r_d, src$ & $=$ & $r_d := r_d ~ \mathrm{xor} ~ src$ \\            
    $\mathrm{jeq ~ r_d, src, +off}$ & $=$ & $\mathrm{if} ~ (r_d = src) ~ \mathrm{then} ~ \mathrm{off}$ \\
    $\mathrm{jneq ~ r_d, src, +off}$ & $=$ & $\mathrm{if} ~ (r_d \neq src) ~ \mathrm{then} ~ \mathrm{off}$ \\
    $\mathrm{jge ~ r_d, src, +off}$ & $=$ & $\mathrm{if} ~ (r_d \geq src) ~ \mathrm{then} ~ \mathrm{off}$    \\
    $\mathrm{jlt ~ r_d, src, +off}$ & $=$ & $\mathrm{if} ~ (r_d < src) ~ \mathrm{then} ~ \mathrm{off}$    \\
    $\mathrm{jle ~ r_d, src, +off}$ & $=$ & $\mathrm{if} ~ (r_d \leq src) ~ \mathrm{then} ~ \mathrm{off}$    \\    
    $\mathrm{ldxdw ~ r_d, [src + \mathrm{off}]}$ & $=$ & $r_d := \mathrm{load(src, off)}$ \\
    $\mathrm{stxdw ~ [r_d + off], src}$ & $=$ & $\mathrm{store(dst, off, src)}$ \\    
  \end{tabular}
  \caption{Translating eBPF assembly to Featherweight eBPF}
\label{fig:translation}
\end{figure}

