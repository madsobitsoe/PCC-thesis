\subsection{Featherweight eBPF}
\label{subsec:featherweight_ebpf}
Will contain definition of the subset of eBPF I will be covering and working with.

% TODO: Should this be in background?
At the beginning of all programs, assume that
\begin{itemize}
\item \texttt{r1} contains \texttt{*ctx}, a pointer to "some memory".
\item \texttt{r2} contains a 64-bit integer indicating the size of the memory pointed to by \texttt{r1}, i.e. reading bytes between \texttt{r1} and \texttt{r1+(r2-1)} are allowed
\item \texttt{r10} contains a frame pointer to 512 bytes of stack-space for the eBPF program
\item \texttt{r0}, \texttt{r3}, \texttt{r4}, \texttt{r5}, \texttt{r6}, \texttt{r7}, \texttt{r8} and \texttt{r9} are all "uninitialized", i.e. they have an unknown value at program start. In "proper eBPF" it is invalid to read an uninitialized register. In x86\_64 assembly, i.e. "jited eBPF", it is valid (but stupid in most cases).

\end{itemize}