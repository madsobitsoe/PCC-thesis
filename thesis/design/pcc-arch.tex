\subsection{Proof Carrying Code Architecture}
\label{subsec:pcc_architecture}

The Proof Carrying Code architecture consists of multiple parts:


\begin{itemize}
\item Featherweight eBPF Weakest Precondition Verification Condition Generator
\item The CVC5 SMT Theorem Prover\cite{gh:cvc5}
\item The LFSC proof certificate checker \texttt{lfscc}\cite{gh:lfsc}
\end{itemize}

When the code producer has written a Featherweight eBPF program, they can generate a verification condition for the program using the \texttt{FWeBPFVCGen}. This will result in a program in SMT-LIB2 format, that represents the verification condition for the program in a standardised format accepted by most SMT solvers.
This can in turn be given to the CVC5 SMT Theorem Prover, which will yield a proof, given that the program does not violate the safety policies, with the caveat that not all problems are solvable using SMT and CVC5 might give up.
The proof will be output in the LFSC-format using the extension \texttt{.plf}.
This proof can in turn be given to the LFSC proof certificate checker \texttt{lfscc}, along with the theories needed for the proof. If the proof is valid, \texttt{lfscc} will output \verb!success! and the program does not violate the safety policies. 




\subsubsection{Shortcomings of the architecture}

While the architecture works fine as a proof of concept and is sufficient to support the work done in this project, it is far from ideal and has many shortcomings.

First of all, the entire process of verification condition generation, proof-generation and proof checking happens in user space. Ideally the proof checking should be performed by the code consumer, i.e. in kernel space, before rejecting or accepting a program. 

The next immediate shortcoming is the multitude of file formats in use. The current pipeline of file formats is $$\mathrm{Assembly} \rightarrow \mathrm{SMTLIB2} \rightarrow \mathrm{LFSC}$$. 

The perhaps gravest shortcoming is that the ``tamper-proof'' property of Proof Carrying Code is lost.
The LFSC proof checker only accepts proofs in \texttt{LFSC} format, and CVC5 only accepts programs in \texttt{SMTLIB2} format, while outputting proofs in \texttt{LFSC} format. The proofs include an encoding of the verification condition, so the verification condition and the proof for it are not decoupled as they should be.

\subsubsection{Proposal for a future architecture}

A solution to the shortcomings listed in the previous section can roughly be sketched out as figure \ref{fig:pcc-architecture}.

Realising this architecture would require a way of decoupling the verification conditions from the proofs as well as significant work in embedding a proof checker in the Linux kernel. This proof checker could for instance be a derivative of \texttt{lfscc}.





% \subsubsection*{Draft diagram of PCC Architecture}

\begin{figure}[htbp!]
  \centering
\inputminted{text}{figures/pcc-architecture.txt}
  \caption{A proposal for a design of a PCC Architecture for eBPF}
  \label{fig:pcc-architecture}
\end{figure}

