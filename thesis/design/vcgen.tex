\subsection{Safety policies for Featherweight eBPF}
\label{subsec:safety_policies_for_featherweight_ebpf}


\begin{itemize}
\item No division by zero. The in-kernel eBPF verifier disallows the immediate constant 0 as divisor, but does not disallow registers used as divisors potentially being 0. 
\item No modulo by zero. 
% \item Memory safety for stack reads and writes.
\item Memory safety for context reads and write.
% \item ``safe'' use of uninitialized registers, such as \texttt{xor r0, r0} to set \texttt{r0 = 0}. Common patterns from x86\_64 assembly that are easy to verify safe but explicitly not allowed in eBPF.
%   (Not sure if this is a safety policy - or can be formulated as one.)
% \item Program termination. (Maybe). \newline
%   Though ``termination'', i.e. the DAG-analysis performed by the verifier seems an obvious candidate, it is trivial as long as FW-eBPF does not include jumps or branching.

\end{itemize}


\subsection{VCGen with WP}
\label{subsec:vcgen_wp}

The predicates that can be generated by weakest precondition verification condition generation of Featherweight eBPF are essentially first-order logic, but extended with an \texttt{if-then-else} construct that can be seen as syntactic sugar for logical exclusive or. The predicates are defined by the grammar
\begin{figure}[H]
  \centering
  \begin{tabular}{lclr}
    $P$ & $::=$ & $ep$ & \\
                 & $|$ & $\neg P$ & (negation) \\    
                 & $|$ & $\forall v . P$ & (Universal quantifier) \\
                 & $|$ &  $P_1 \land P_2$ & (Logical conjunction) \\
                 & $|$ &  $P_1 \implies P_2$ & (Implication) \\
                 & $|$ &  \texttt{if} $ep$ \texttt{then} $P_1$ \texttt{else}  $P_2$ & (Logical exclusive or) \\        
\end{tabular}    
\end{figure}



The rules for WP computation can be seen in figure \ref{fig:wp-rules}. They are heavily inspired by the WP-rules presented in \cite{DAILLER201897}.

\begin{figure}[H]
  \centering
\[
  \mathrm{WP}(\mathcal{P}, idx, Q) =
  \begin{cases}
    \mathrm{true}, & \mathcal{P}_{idx} = \mathrm{exit} \\
    \forall v . v = p \implies \mathrm{WP}(\mathcal{P}, idx+1, Q)[x \leftarrow v] , & \mathcal{P}_{idx} = x := p \\    
    \forall v . v = p_1 \oplus p_2 \implies \mathrm{WP}(\mathcal{P}, idx+1, Q)[x \leftarrow v] , & \mathcal{P}_{idx} = x := p_1 \oplus p_2 \\
    \forall v . v = p_1 \ominus p_2 \implies \mathrm{WP}(\mathcal{P}, idx+1, Q)[x \leftarrow v] , & \mathcal{P}_{idx} = x := p_1 \ominus p_2 \\    
    \forall v . v = p_1 \otimes p_2 \implies \mathrm{WP}(\mathcal{P}, idx+1, Q)[x \leftarrow v] , & \mathcal{P}_{idx} = x := p_1 \otimes p_2 \\    
    \forall v . (p_2  \neq 0 \land (v = p_1 \oslash p_2 \implies \mathrm{WP}(\mathcal{P}, idx+1, Q)[x \leftarrow v])) , & \mathcal{P}_{idx} = x := p_1 \oslash p_2 \\
\forall v . (p_2  \neq 0 \land (v = p_1 ~ \mathrm{mod} ~ p_2 \implies \mathrm{WP}(\mathcal{P}, idx+1, Q)[x \leftarrow v])) , & \mathcal{P}_{idx} = x := p_1 ~ \mathrm{mod} ~ p_2 \\    
    \forall v . v = p_1 ~ \mathrm{xor} ~ p_2 \implies \mathrm{WP}(\mathcal{P}, idx+1, Q)[x \leftarrow v] , & \mathcal{P}_{idx} = x := p_1 ~ \mathrm{xor} ~ p_2 \\
    \mathrm{if} ~ p = e ~ \mathrm{then} ~ \mathrm{WP}(\mathcal{P}, idx+1+\mathrm{off}, Q) ~ \mathrm{else} ~ \mathrm{WP}(\mathcal{P},idx+1, Q) , & \mathcal{P}_{idx} = \mathrm{if} ~ p = e ~ \mathrm{then} ~ \mathrm{\mathrm{off}} \\
    \mathrm{if} ~ p \neq e ~ \mathrm{then} ~ \mathrm{WP}(\mathcal{P}, idx+1+\mathrm{off}, Q) ~ \mathrm{else} ~ \mathrm{WP}(\mathcal{P},idx+1, Q) , & \mathcal{P}_{idx} = \mathrm{if} ~ p \neq e ~ \mathrm{then} ~ \mathrm{\mathrm{off}} \\    
    \mathrm{if} ~ p \geq e ~ \mathrm{then} ~ \mathrm{WP}(\mathcal{P}, idx+1+\mathrm{off}, Q) ~ \mathrm{else} ~ \mathrm{WP}(\mathcal{P},idx+1, Q) , & \mathcal{P}_{idx} = \mathrm{if} ~ p \geq e ~ \mathrm{then} ~ \mathrm{\mathrm{off}} \\
    \mathrm{if} ~ p \leq e ~ \mathrm{then} ~ \mathrm{WP}(\mathcal{P}, idx+1+\mathrm{off}, Q) ~ \mathrm{else} ~ \mathrm{WP}(\mathcal{P},idx+1, Q) , & \mathcal{P}_{idx} = \mathrm{if} ~ p \leq e ~ \mathrm{then} ~ \mathrm{\mathrm{off}} \\
    \mathrm{if} ~ p < e ~ \mathrm{then} ~ \mathrm{WP}(\mathcal{P}, idx+1+\mathrm{off}, Q) ~ \mathrm{else} ~ \mathrm{WP}(\mathcal{P},idx+1, Q) , & \mathcal{P}_{idx} = \mathrm{if} ~ p < e ~ \mathrm{then} ~ \mathrm{\mathrm{off}} \\
    \begin{aligned}[b] \forall v , v' . & (\mathrm{off} \geq 0 \land \mathrm{off} \leq (n \ominus 8) ) \\
                                        & \land (0 = \mathrm{off} ~ \mathrm{mod} ~ 8)
      \\ & \land v = v' \implies \mathrm{WP}(\mathcal{P}, idx+1, Q)[x \leftarrow v] \end{aligned}  , & \mathcal{P}_{idx} = r_d := \mathrm{load(src, off)} \\
(\mathrm{off} \geq 0 \land \mathrm{off} \leq (n \ominus 8) ) \land (0 = \mathrm{off} ~ \mathrm{mod} ~ 8)
\land \mathrm{WP}(\mathcal{P}, idx+1, Q)  , & \mathcal{P}_{idx} = \mathrm{store(dst, off, src)} \\
  \end{cases}
\]

  \caption{The rules for WP computation for Featherweight eBPF.}
  \label{fig:wp-rules}
\end{figure}



After WP computation for a given program has resulted in a verification condition $Q$, the following \textit{precondition} is prepended:

$$
\forall n . n \geq 8 \implies Q
$$
