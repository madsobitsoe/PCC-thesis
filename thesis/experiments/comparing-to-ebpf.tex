\subsection{Comparing Featherweight eBPF and eBPF}
\label{subsec:comparing_featherweight_ebpf_and_ebpf}

As Featherweight eBPF by definition is a small subset of eBPF, that adds restrictions wrt. to the calling convention of programs, a direct comparison between the two is not possible.
However, in order to give a valuable assessment of the proof carrying code implementation for Featherweight eBPF, a comparison is needed.


eBPF follows the \texttt{x86\_64} calling convention, st. \texttt{r0} is used for the return value, \texttt{r1, r2, r3, r4, r5} are used for arguments to function calls\footnote{\url{https://docs.kernel.org/bpf/instruction-set.html\#registers-and-calling-convention}}.
From this follows that when eBPF programs are executed, \texttt{r1} will contain a pointer to ``context''\footnote{
  \url{https://docs.kernel.org/bpf/verifier.html}}.

This pointer in \texttt{r1} \textit{might} point to a structure that contains the size of the memory area, but it depends on the type of the program\footnote{This is not really documented outside of the kernel sources, but this blog post at least gives a brief overview of some types of programs:
  \url{https://blogs.oracle.com/linux/post/bpf-a-tour-of-program-types}}.
In Featherweight eBPF I assume that \texttt{r1} contains a pointer to ``context'' and further that \texttt{r2} contains the size of that memory area that is valid to read from.

Luckily eBPF maps allows us to mimic the restrictions in Featherweight eBPF, by prepending a short header to all programs before checking if the eBPF verifier will allow the program to run.

The setup is as follows:

\begin{itemize}
\item Reserve register \texttt{r9}, st. it is never used in programs. It will solely be used for setting up the eBPF program.
  \item Also ``reserve'' register \texttt{r0}. As it is used for return values from kernel helper functions, it is needed for setting up the eBPF program. This will cause the in-kernel verifier to range-track the register before the actual program we want to verify, and we want to have a ``clean verifier'' for the registers used by our program.
\item let \texttt{sizemap} be an eBPF array map with a single entry of size 8. This will hold a 64-bit integer $n$.
\item let \texttt{ctxmap} be an eBPF array map of with a single entry of size $n$. This will represent the context given to a Featherweight eBPF program.
\item Load the map file descriptor for \texttt{sizemap}, then load the value it contains into \texttt{r2}.
  \item Load the map file descriptor for \texttt{ctxmap} into \texttt{r1}. 
  \end{itemize}
  This will remove any reference to the actual ``context'' of the eBPF program, and ensure the rest of the program will run with the same assumptions as the Featherweight eBPF equivalent.

  
  After the header has been prepended, we can attempt to load the program using the \texttt{bpf} syscall. This will trigger the in-kernel verifier to analyse and verify the program. If the program verifies and is allowed to run, we get a valid file descriptor in return, if the program does not verify, we get an invalid file descriptor.
  We are not interested in actually running the program, only whether it verifies or not.

In all experiments the constant $n$ is set to 64, meaning $r1$ will contain a pointer to 64 bytes, or 8 consecutive 8-byte values. Since the constant $n$ is loaded from the shared map at runtime, the in-kernel verifier has no way of knowing the value. However, since the two maps have to be created before loading the eBPF program, the in-kernel verifier will know the size of \texttt{ctxmap} and use it for e.g. bounds checking. 
  
  The header, written using ebpf-tools, can be seen in listing \ref{snip:compare-header}.

  \begin{figure}[ht!]
    \centering
\inputminted[linenos]{haskell}{snippets/compare-header.hs}
    \caption{The header that is prepended to Featherweight eBPF programs before attempting to verify them using the in-kernel verifier.}
    \label{snip:compare-header}
  \end{figure}



