\subsection{Program Collection}
\label{subsec:program_collection}

% Will contain the primarily synthetic programs I will be working with and basing Featherweight eBPF on.
% \newline
% For each program:
% \begin{itemize}
%     \item Why is the program relevant and what does it show/test/exercise?  
%     \item Does it verify with eBPF verifier?  (TODO: How do I translate FW-eBPF into ``real'' eBPF?)
%     \item Can CVC5 prove it safe?
% \end{itemize}

In Table \ref{tbl:program-collection} an overview of the programs included in the program collection can be found. The program collection constitute the entire ``test suite'' for comparing Featherweight eBPF to the in-kernel eBPF verifier.

\begin{table}[!ht]
\begin{tabularx}{\textwidth}{|>{\hsize=1.25\hsize}X|>{\hsize=.75\hsize}X|>{\hsize=.75\hsize}X|>{\hsize=1\hsize}X|>{\hsize=1\hsize}X|>{\hsize=1.25\hsize}X|} \hline
  \textbf{Program name} & \textbf{Reference} & \textbf{Is it safe?} & \textbf{Verifies with eBPF verifier} & \textbf{CVC5 can prove it safe} & \textbf{Relevant safety policies} \\ \hline
  Exit immediately & \ref{prog:justexit} & \checkmark & \checkmark  & \checkmark & None \\ \hline  
  Potential Division By Zero & \ref{prog:bad_div} & No. & \checkmark & No & No divide by zero \\ \hline
  Trivial Safe Division & \ref{prog:good_div} & \checkmark & \checkmark & \checkmark & No division by zero \\ \hline
  Safe Division By Means Of Precondition & \ref{prog:good_div_pre} & \checkmark & \checkmark & \checkmark & No division by zero \\ \hline

  % XOR Initialisation & \ref{prog:xor_init} & ? & ? & ? & Safe use of uninitialised registers  \\ \hline
  Maybe Load And Sum Four 8-Byte chunks & \ref{prog:load4times_safe} & \checkmark & \checkmark & \checkmark & Memory safety for context reads. \\ \hline
Xor Leads To Unconditional Jump & \ref{prog:xor_jump} & \checkmark & No.  & \checkmark & No division by zero. \\ \hline  
\end{tabularx}
\caption{An overview of the programs in the program collection, their relevance and what safety policies they showcase.}
\label{tbl:program-collection}
\end{table}

\subsubsection{Exit immediately}
\label{prog:justexit}
This program is included mainly as a sanity check. It is the shortest possible program that can be verified in eBPF.

\inputminted[linenos]{asm}{programs/justexit.asm}

\subsubsection{Potential Division By Zero}
\label{prog:bad_div}

This program contains two divisions. The result of the first division in line 2 might be 0. The result is then used as divisor in line 3.
The in-kernel verifier allows the program to run, as potential division by 0 is patched with a runtime conditional jump in eBPF. Given the generated verification condition, CVC5 can find a satisfying assignment, meaning we can prove division by 0 might happen and reject the program as it violates the safety policy. 
\inputminted[linenos]{asm}{programs/bad_div.asm}


\subsubsection{Trivial Safe Division}
\label{prog:good_div}

In line 2 register \texttt{r1} is set to the value 2. \texttt{r1} is then used as divisor. As $2 \neq 0$, this division is trivially safe.

\inputminted[linenos]{asm}{programs/good_div.asm}

\subsubsection{Safe Division by means of preconditions}
\label{prog:good_div_pre}

In line 2 register \texttt{r2} is used as divisor. As the precondition for all programs states that $r2 \geq 8$ this division is safe.

\inputminted[linenos]{asm}{programs/good_div_pre.asm}


\subsubsection{Maybe Load And Sum Four 8-byte Chunks}
\label{prog:load4times_safe}

The program starts with a conditional jump based on the memory size. If it is possible to read 4 consecutive 8-byte chunks from memory, the program will sum those chunks. If it is not possible, the program simply exits without reading memory. 

\inputminted[linenos]{asm}{programs/load4times_safe.asm}


\subsubsection{Xor leads to unconditional jump}
\label{prog:xor_jump}
This program is admittedly a contrived example, designed such that the in-kernel verifier rejects the program while we can prove it safe. 
The program uses an \texttt{xor} operation to ensure a register is 0, after which a conditional jump is based on that register being 0. In the non-zero case, the program will perform a divison by zero and violate a safety policy, then exit without initialising \texttt{r0}, violating an eBPF requirement.
The program is rejected by the in-kernel verifier, as it does not know the conditional jump will always be taken.



\inputminted[linenos]{asm}{programs/xor_jump.asm}


% \subsubsection{XOR initialisation}
% \label{subsubsec:xor_init}

% \inputminted[linenos]{text}{programs/xor_init.ebpf}


