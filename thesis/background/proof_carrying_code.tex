\subsection{Proof Carrying Code}
\label{subsec:proof_carrying_code}

Will cover learning objective ``Explain what Proof Carrying Code is and explain the architecture behind''.

\begin{itemize}
\item 
Code consumer vs. code producer.
\item VC and VCGen
  \item Architecture with VCGen ensures proof is tamper-proof, as VC are extracted by proof checker.
\end{itemize}




\subsubsection{Safety Policies}
\label{subsec:safety_policies}
\begin{itemize}
\item Examples of safety policies
\item safety vs. security
\end{itemize}
% \begin{itemize}
% \item No divide by zero. (Not in the verifier)
% \item Memory safety for stack reads and writes.
% \item Memory safety for context reads.
% % \item ``safe'' use of uninitialized registers, such as \texttt{xor r0, r0} to set \texttt{r0 = 0}. Common patterns from x86\_64 assembly that are easy to verify safe but explicitly not allowed in eBPF.
% %   (Not sure if this is a safety policy - or can be formulated as one.)
% % \item Program termination. (Maybe). \newline
% %   Though ``termination'', i.e. the DAG-analysis performed by the verifier seems an obvious candidate, it is trivial as long as FW-eBPF does not include jumps or branching.

% \end{itemize}

