\subsection{Proof Carrying Code}
\label{subsec:proof_carrying_code}

Will cover learning objective ``Explain what Proof Carrying Code is and explain the architecture behind''.

\begin{itemize}
\item 
Code consumer vs. code producer.
\item VC and VCGen
  \item Architecture with VCGen ensures proof is tamper-proof, as VC are extracted by proof checker.
\end{itemize}




\subsubsection{Safety Policies}
\label{subsec:safety_policies}

In Proof Carrying Code a safety policy is a definition of properties that needs to be proved about a program, in order to consider the program safe. An example of such a policy is ``no division by zero'', which would require that for all divisions in the program, a proof that the divisor can not be 0 at runtime. Another example of a safety policy is bounds-checking, i.e. that all memory accesses are in-bounds. This safety policy can take many forms depending on the language, but in general requires the memory size(s) to be known at the time of proof generation, or at least be available during runtime st. a proof of proper runtime bounds-checking can be generated. 


\textit{Safety} policies differ from \textit{security} policies. Examples of security policies are e.g. ensuring no use of uninitialised registers, ensuring a program has no unreachable paths, ensuring a program does not leak kernel pointers to user space, no type mismatches etc. In this project I only explicitly care about safety policies, though some hints of security policies inevitably sneaks in. 


% \begin{itemize}
% \item Examples of safety policies
% \item safety vs. security
% \end{itemize}
% \begin{itemize}
% \item No divide by zero. (Not in the verifier)
% \item Memory safety for stack reads and writes.
% \item Memory safety for context reads.
% % \item ``safe'' use of uninitialized registers, such as \texttt{xor r0, r0} to set \texttt{r0 = 0}. Common patterns from x86\_64 assembly that are easy to verify safe but explicitly not allowed in eBPF.
% %   (Not sure if this is a safety policy - or can be formulated as one.)
% % \item Program termination. (Maybe). \newline
% %   Though ``termination'', i.e. the DAG-analysis performed by the verifier seems an obvious candidate, it is trivial as long as FW-eBPF does not include jumps or branching.

% \end{itemize}

