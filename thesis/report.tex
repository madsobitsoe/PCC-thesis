\documentclass[a4paper,10pt]{article}

% Packages
\usepackage[utf8]{inputenc}
\usepackage[T1]{fontenc}
%\usepackage[danish]{babel}
\usepackage{hyperref}
\usepackage{graphicx}
\usepackage{amsmath}
\usepackage{subcaption}
\usepackage{minted}
\usepackage{xcolor}
\usepackage{tikz}
\usepackage{pgfplots}
\usepackage{tabularx}
\usepackage{stmaryrd}
\usepackage{bussproofs}
\graphicspath{{images/}}
\pgfplotsset{compat=1.13}

% Basic layout:
\setlength{\textwidth}{175mm}
\setlength{\textheight}{250mm}
\setlength{\parindent}{0mm}
\setlength{\parskip}{\parsep}
\setlength{\headheight}{0mm}
\setlength{\headsep}{5mm}
\setlength{\hoffset}{-10mm}
\setlength{\voffset}{-15mm}
\setlength{\footskip}{15mm}
\setlength{\oddsidemargin}{0mm}
\setlength{\topmargin}{0mm}
\setlength{\evensidemargin}{0mm}

% Has to be included after messing with layout
\usepackage{thesis-style}

% Blackboard bold
\newcommand{\NN}{\mathbb{N}}
\newcommand{\ZZ}{\mathbb{Z}}
\newcommand{\QQ}{\mathbb{Q}}
\newcommand{\RR}{\mathbb{R}}
\newcommand{\EE}{\mathbb{E}}
\newcommand{\PP}{\mathbb{P}}
\newcommand{\FF}{\mathbb{F}}
\newcommand{\CC}{\mathbb{C}}

% Declare
\DeclareMathOperator{\sinc}{sinc}

% Colors
\definecolor{bblue}{HTML}{4F81BD}
\definecolor{rred}{HTML}{C0504D}
\definecolor{ggreen}{HTML}{9BBB59}
\definecolor{ppurple}{HTML}{9F4C7C}

% Misc
\date{\today}
\author{Mads Obitsø Thomsen <scr411>}

\setStaticDate{\today}

\setAuthors{Mads Obitsø Thomsen <scr411>}
\setShortAuthors{Obitsø}


\setAssignment{Masters Thesis - eBPF and PCC}
\title{\course\small{DIKU, 2022}}

\begin{document}
\maketitle

\hrulefill

\section{Introduction}
\label{sec:introduction}

\section{Background}
\label{sec:background}

\subsection{Crash course in the linux kernel architecture (brief)}
\label{subsec:crash_course_in_the_linux_kernel_architecture_(brief)}
Brief section giving a quick overview of the possibilities of extending the linux kernel, i.e. new kernel, new module or eBPF.


\subsection{eBPF Subsystem}
\label{subsec:ebpf_subsystem}
Will cover learning objective ``Explain the eBPF subsystem, in particular the machine- and programmign model''.

\subsubsection{eBPF Machine Model}
\subsubsection{eBPF Programming Model}
\subsubsection{eBPF Verifier - actions and guarantees}
Will cover learning objective ``Explain the functionality of the eBPF verifier and what it guarantees''.
Will cover learning objective ``Analyze and identify problems related to untrusted code (and sandboxing?)''
\subsection{Proof Carrying Code}
\label{subsec:proof_carrying_code}

Will cover learning objective ``Explain what Proof Carrying Code is and explain the architecture behind''.

\section{Implementation}
\label{sec:implementation}

Will cover learning objective ``Modify and extend the linux kernel, in particular the eBPF subsystem.''
Will cover learning objective ``Write C-code for the linux kernel adhering to current guidelines and standards.''










\bibliographystyle{splncs04}
\bibliography{references}

\appendix
\end{document}
